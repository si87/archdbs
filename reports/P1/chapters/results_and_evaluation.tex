\chapter{Ergebnisse und Auswertung}
\section{Datenbankgröße}
Der gemessene \texttt{Footprint} unserer leeren Datenbank \texttt{ArchDBS} war 128 MB. Bei genauerer Betrachtung setzt sich die leere Datenbank wie folgt zusammen:\\

--ToDo Table pagesize--\\

Die Seitengröße beträgt für alle drei Tablespaces 4096 Byte.\\

Nach Abruf der Informationen mittels \texttt{list tablespaces show detail} wuchs die Größe der Datenbank allerdings auf 160 MB an. Gründe dafür könnten der Verwaltungsaufwand der Datenbank zur Ermittlung der Tabellendetails sein.\\

Nach der Erzeugung der Testdaten war die Datenbank 192 MB groß. Unsere errechnete Größe lag allerdings bei 183 MB (160 MB + 13.5 MB). Der Fehler beträgt somit unter fünf Prozent. Die Differenz ist hat (hauptsächlich) zwei Ursachen, welche daraus resultieren, dass wir lediglich die genaue Größe der reinen Datenmenge berechnet haben, aber nicht den Overhead eingerechnet haben. Zum einen werden noch die drei Indexe für \texttt{Kunde}, \texttt{Produkt} und \texttt{Bestellung} (mit 750000 Datensätzen) in der Datenbank angelegt, zum anderen werden auch die Seiten nicht zu genau 100 Prozent befüllt. Wenn der Datensatz nicht mehr auf eine Seite passt, wird der freie Restspeicherplatz leer gelassen und die nächste Seite erzeugt. Dadurch beanspruchen die Daten insgesamt mehr Speicherplatz als in unserer Rechnung.\\

\section{Zeitliche Messungen}
Folgende Tabelle stellt die gemessenen Werte dar:\\
--ToDo Tabelle mit Werten--\\

In dieser Tabelle ist zu sehen, dass das Eintragen in die Tabellen der Datenbank \texttt{ArchDBS} wesentlich länger dauert.