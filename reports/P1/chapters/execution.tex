\chapter{Durchführung}
Über das Befehlszeilenfenster wurde die Datenbank \texttt{ArchDBS} erzeugt und den \textit{Footprint} der leeren Datenbank ermittelt. Danach wurde das Java-Konsolen-programm erstellt, das 100 Kundentupel, 1000 Produkttupel und 750 000 Bestellungen in die Datenbank eingetragen hat. Dabei wurde die Zeit gemessen, die für die Erstellung der 751100 Tupel benötigt wurden. Damit die Erzeugung der Testdaten performant ablaufen konnte, wurde \texttt{autocommit} auf \texttt{false} gesetzt und \texttt{PreparedStatements} verwendet. Ebenso war es nötig, die maximale Größe der Logdateien (\texttt{LOGPRIMARY} oder \texttt{LOGSECOND}) zu erhöhen, damit die Erzeugung der 750000 Bestellungen auch mitgeloggt werden können. Die Größe der Datenbank wurde erneut ermittelt und mit dem Startwert verglichen.\\

Im nächsten Schritt wurde eine zweite Datenbank mit den entsprechenden Tabellen angelegt, wobei auf Foreign-Keys-Indexe verzichtet worden sind. Erneut wurde die Zeit gemessen und mit dem Wert aus der vorherigen Messung verglichen. Als letztes wurden die Fremdschlüsselindexe über die gefüllten Tabellen erzeugt. 