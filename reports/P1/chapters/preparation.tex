\chapter{Vorbereitung}
Im Laufe des erstens Praktikums muss die Datenbank mit Testdaten befüllt werden. Im Vorfeld soll dabei die Größe der Datenbank geschätzt werden, nachdem die Daten erzeugt und gespeichert worden sind. 
Dabei wurde mit Hilfe folgender Tabelle [ToDo:cite] die Größe eines Datensatzes ermittelt und mit der Anzahl der zu schreibenden Datensätze multipliziert.

Die folgende Auflistung zeigt die Hochrechnung für die drei Tabellen \texttt{Kunde}, \texttt{Produkt} und \texttt{Bestellung} an.\\

\begin{tabular}{|l|l|r|} \hline\hline
\multicolumn{3}{|l|}{ \textbf{Kunde}}\\
Spaltenname & Typ & Byte Count\\
\hline 
KNR & Integer & 4 Byte\\
KNAME & Char & 20 Byte \\ 
KVORNAME & Char & 15 Byte \\ \hline
\multicolumn{2}{|r}{} & 100 x 39 Byte\\
\multicolumn{2}{|r}{} & = 3900 Byte \\ \hline
\end{tabular}


\begin{tabular}{|l|l|r|} \hline\hline
\textbf{Typ} & \multicolumn{2}{c|}{\textbf{Stil}}\\
\hline
ausgekocht & pink & klein\arrayrulewidth2pt\vline\\
\cline{2-3}
ziemlich d"amlich & purpur & gro"s \\ \hline\hline
\end{tabular}

Als Gesamtgröße für die reine Datenmenge wird in etwa 13,5 MB angenommen (bei 1000Byte = 1 KByte).\\

Ein weiteres Ziel dieses Praktikums ist, ein Gefühl für die Eintragungsdauer großer Datenmengen zu bekommen und worin die Zeitunterschiede begründet sind.