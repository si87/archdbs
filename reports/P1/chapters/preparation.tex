\chapter{Vorbereitung}
Im Laufe des erstens Praktikums muss die Datenbank mit Testdaten befüllt werden. Im Vorfeld soll dabei die Größe der Datenbank geschätzt werden, nachdem die Daten erzeugt und gespeichert worden sind. 
Dabei wurde mit Hilfe der Tabelle (siehe Anhang \ref{app:01}) die Größe eines Datensatzes ermittelt und mit der Anzahl der zu schreibenden Datensätze multipliziert.\\

Die folgende Auflistung zeigt die Hochrechnung für die drei Tabellen \texttt{Kunde}, \texttt{Produkt} und \texttt{Bestellung} an (siehe Tabelle \ref{tbl:size_of_data}).
\begin{longtable}{|l|l|r|r|} \hline
%\begin{tabular}{|l|l|r|l|} 
\multicolumn{4}{|l|}{ \textbf{Kunde}}\\
Spaltenname & Typ & Byte Count & \\
\hline 
KNR & Integer & 4 Byte & \\
KNAME & Char & 20 Byte & \\ 
KVORNAME & Char & 15 Byte &\\ \hline
\multicolumn{2}{|r}{} & 100 x 39 Byte & = 3 900 Byte \\ \hline

\multicolumn{4}{|l|}{ \textbf{Produkt}}\\
Spaltenname & Typ & Byte Count &\\
\hline 
PID & Integer & 4 Byte & \\
PRODUKTNAME & Char & 20 Byte & \\ 
PREIS & Decimal(6,2) & 4 Byte & \\ \hline
\multicolumn{2}{|r}{} & 1000 x 28 Byte & = 28 000 Byte \\ \hline

\multicolumn{4}{|l|}{ \textbf{Bestellung}}\\
Spaltenname & Typ & Byte Count &\\ \hline 
BID & Integer & 4 Byte & \\
KNR & Integer & 4 Byte & \\ 
PID & Integer & 4 Byte & \\ 
DATUM & Date & 4 Byte & \\ 
STATUS & Smallint & 2 Byte & \\ \hline
\multicolumn{2}{|r}{} & 750 000 x 18 Byte & = 13 500 000 Byte \\ \hline
\multicolumn{3}{|r}{Summe} & = 13 531 900 Byte \\ \hline
%\end{tabular}
\caption{Berechnung der Größe der Testdaten}
\label{tbl:size_of_data}
\end{longtable}
Als Gesamtgröße für die reine Datenmenge wird in etwa 13,5 MB angenommen (bei 1000Byte = 1 KByte). 

\section*{Fremdschlüsselindex}
Wird der Fremdschlüsselindex nicht angelegt bei der Erzeugung der Tabelle, sollte das Eintragen der Daten schneller gehen. Nach unseren Erwartungen sollte das nachträgliche Erzeugen des Fremdschlüsselindexes für die Tabelle \texttt{Bestellung} einige Zeit in Anspruch nehmen.