\chapter{Durchführung}
Als erstes wurde die automatische Leistungsoptimierung der DB2-Datenbank abgeschaltet, um reproduzierbare Ergebnisse zu erhalten. Danach wurde Szenario 1.1 implementiert. Dabei wurde nur mit der Klasse \texttt{Statement} gearbeitet. Erst in Szenario 1.2 wurden \texttt{PreparedStatement}s verwendet. Das Lesen von 300er Portionen wurde anhand einer Zählvariable realisiert, die in jedem Durchlauf um 300 erhöht und in der \texttt{WHERE}-Klausel angegeben wird. Dabei machen wir uns zu Nutze, dass wir beim Eintragen der Bestellungsdaten für die ID eine fortlaufende Sequenz verwenden.

Szenario 2?

Für das dritte Szenario wurden immer nur die ersten 300 Bestellungsdatensätze gelesen. Für das vierte Szenario wurden die Datensätze in 10er-Tupel gelesen. Dabei wurden die verschachtelten Schleifen angepasst.