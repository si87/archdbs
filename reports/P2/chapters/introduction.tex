\chapter{Einleitung}
Im letzten Praktikum haben wir uns auf einer sehr tiefen Ebene mit der DB2-Datenbank beschäftigt, indem wir die Byte-Größe der Indexe und der Daten betrachtet haben. In diesem Praktikum werden die Daten aus dem vorherigen Praktikum als Testdaten verwendetet, um verschiedene Anwenderprofile zu simulieren und deren Auswirkung auf die Performance zu beurteilen. \\

Dafür sollen folgende vier Szenarios durchgespielt werden:
\begin{description}
\item[Szenario 1.1] Viermaliges sequenzielles lesen aller Bestellungsdatensätze in 300er-Portionen mit der Klasse \texttt{Statement}
\item[Szenario 1.2] Viermaliges sequenzielles lesen aller Bestellungsdatensätze in 300er-Portionen mit der Klasse \texttt{PreparedStatement}
\item[Szenario 2] Lesen von 3.000.000 Bestellungsdatensätze in 300er-Portionen in zufälliger Reihenfolge.
\item[Szenario 3] Lesen von 3.000.000 Bestellungsdatensätze, wobei immer die selben 300 Datensätze gelesen werden.
\item[Szenario 4] Wie Szenario 1, nur diesmal in 10er-Portionen
\end{description}

Szenario 1.1 und 1.2 dienen dabei als Referenzgröße. Es sollen bei allen Szenarios sichergestellt werden, dass alle bzw. die richtigen Datensätze gelesen werden können. 